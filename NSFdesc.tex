
%%%%%%%%% PROPOSAL -- 15 pages (including Results from Prior NSF Support)

\required{Project Description}

% From the NSF Grants Proposal Guide:
% "The Project Description should provide a clear statement of the work 
% to be undertaken and must include the objectives for the period of the 
% proposed work and expected significance; the relationship of this work
% to the present state of knowledge in the field, as well as to work in 
% progress by the PI under other support.
%
% The Project Description should outline the general plan of work, 
% including the broad design of activities to be undertaken, and, 
% where appropriate, provide a clear description of experimental 
% methods and procedures. Proposers should address what they 
% want to do, why they want to do it, how they plan to do it, how 
% they will know if they succeed, and what benefits could accrue
% if the project is successful. The project activities may be based
% on previously established and/or innovative methods and approaches,
% but in either case must be well justified. These issues apply to 
% both the technical aspects of the proposal and the way in which
% the project may make broader contributions."

\section{Mining Pre-Exposure Prophylaxis Trends in Social Media}

Propose research on PrEP and what additional steps can be taken and how you will know if you succeed and how this benifits society etc..

\section{Arvind's secion (to be decided)}


\section{Broad application of biosurveillance and social media patient feedback}

\required{Broader Impacts}
% As in the project summary, broader impacts of the proposed work
% must be called out separately in the project description.  
% You may be able to give more specific examples, 
% or discuss how you've previously achieved these impacts.
% It should be similar, but not identical, to the Broader Impacts statement
% in the project summary.