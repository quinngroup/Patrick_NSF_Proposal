
%%%%%%%%% SUMMARY -- 1 page, third person
% e.g:  "The PI will prove" not "I will prove"

\required{Project Summary}
%\required{Overview}
% This should be a brief description of the activity 
% that would result if the proposal were funded 
% and a statement of objectives and methods to be employed.
% It should look something like an abstract. 

%Biosurveillance is a new interdisciplinary collaboration between public health and data science that has been developed in the last few decades. It leverages a variety of data mining algorithms performed on large data repositories to identify and monitor disease outcomes and treatments.

\required{Mining Pre-exposure Prophylaxis trends in social media}
One specific application will focus on the use of antiretroviral therapy to preventatively treat individuals who are at-risk for HIV infection. This use of antiretroviral therapy for preventative treatment is referred to as Pre-Exposure Prophylaxis (PrEP) and uses the small molecule nucleotide reverse transcriptase inhibitor trade named Truvada. PrEP is highly effective at preventing HIV, but because it is a new treatment there remain some social and medical obstacles preventing it from achieving its full potential. In this section we use various supervised and unsupervised text mining techniques to identify relevant keywords, hashtages, positive tweets and negative tweets in Twitter data that can be used to gain insight into the social concerns surrounding PrEP adoption. By addressing these concerns, we can make PrEP more effective and improve HIV prevention efforts.

\required{Combining Social Media and Phylogenetic information to infer HIV outbreak dynamics}
One of our goals to better understand HIV outbreak dynamics relies on the ability to combine hard medical data and social media data to monitor and predict social media analyses. In this section we propose work similar to previous work conducted in the San Diego regional health system, in which we infer an HIV pylogenetic transmission network from anonymity genetic medical data in the northern Georgia health system, and associate this with geolocation data taken from social media over the same time period. If successful, this project would measure the network of HIV transmission in the northern Georgia health system, link these dynamics to keywords and metadata present in social media data geotagged to the same locations, and determine the regional risk of drug resistant serotypes.


\required{Determining associations between personality and drug perception on social media }
Our previous work has uncovered complex social sentiments, favorable and unfavorable, to the PrEP drug Truvada. Our work and other work suggests that these social sentiments that may be driving adoption of and adherence to Truvada. Personality, commonly measured with the use of the Big Five profile, has been found to be predictive of thoughts and behaviors. Other researchers have used personality to predict disease outbreaks using social media, and adherence to drugs using traditional survey data. However, we have found that the direct association of personality and drug perceptions using social media data is lacking. Here, we suggest a project that would investigate the association between personality and perceptions of drugs from three major categories, psychiatric drugs, pain killers, and HIV related drugs including Truvada. By studying these associations, we can determine social concerns and barriers, that while not directly related to the medical functionality of these drugs, are  critical to ensure that preventable and treatable diseases are optimally combated.