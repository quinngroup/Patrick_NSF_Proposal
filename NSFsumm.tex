
%%%%%%%%% SUMMARY -- 1 page, third person
% e.g:  "The PI will prove" not "I will prove"

\required{Project Summary}
\required{Overview}
% This should be a brief description of the activity 
% that would result if the proposal were funded 
% and a statement of objectives and methods to be employed.
% It should look something like an abstract. 

Human Immunodeficiency Virus (HIV) is a disease that affects millions of individuals in the United States. While HIV has been studied for decades and preventative strategies have been in use for decades, its continued persistence has demonstrated the need for new methods both for both monitoring and prevention. Biosurveillance is a new interdisciplinary collaboration between public health and data science that has been developed in the last few decades. It leverages a variety of data mining algorithms performed on large data repositories to identify and monitor disease outcomes and treatments. In this document the PIs describe an approach to identify trends in disease treatment by applying existing and novel machine learning techniques in the field of biosurveillance. The PI's contribution is both in the development of data analysis pipelines for general purpose biosurveilence, and the application of those pipelines for the study of HIV and mental disease monitoring.

Data sources will include text, images, and metadata associated with social media, and will eventually link these data with direct medical data. High level information, analytical tools, and processed data produced through these analyses will be disseminated back into both clinical and social media venues to improve patient's medical outcomes and address any public misconceptions. While we will collaborate with public health and psychology contributors, our research primarily focuses on the data science side aspects of the broader biosurveillance field.

Our efforts in biosurveillance will be addressed in three sections:

\required{Section1: Mining Pre-exposure Prophylaxis trends in social media}
One specific application will focus on the use of antiretroviral therapy to preventatively treat individuals who are at-risk for HIV infection. This use of antiretroviral therapy for preventative treatment is referred to as Pre-Exposure Prophylaxis (PrEP) and uses the small molecule nucleotide reverse transcriptase inhibitor trade named Truvada. PrEP is highly effective at preventing HIV, but because it is a new treatment there remain some social and medical obstacles preventing it from achieving its full potential. In this section we use various supervised and unsupervised text mining techniques to identify relevant keywords, hashtages, positive tweets and negative tweets in Twitter data that can be used to gain insight into the social concerns surrounding PrEP adoption. By addressing these concerns, we can make PrEP more effective and improve HIV prevention efforts.

\required{Section2: Semi supervised learning and operations on arbitrary dimension tensors}
In order to facilitate our biosurveillance work, we will develop and implement useful computational methods that can run on heterogeneous hardware. Graphical Processing Units (GPUs) have a theoretical computing output at least an order of magnitude more than a similarly priced CPU. By adopting GPU computing hardware we may be able to scale up existing data mining analyses, and still have them run in a reasonable amount of time. One such example, which would be relevant to text mining and biosurveillance in general, would be a higher order version of the word2vec algorithm implemented for the GPU. In this section we propose this and other similar analyses.


\required{Section3: Using machine learning to determine user personality types associated with disease and treatment responses }

Previous research implies that personality type plays an important role in social media interaction dynamics. Specifically, narcissistic personalities tend to be over represented in the highly connected nodes of a social network. In order to understand the social issues surrounding disease treatment, it would be helpful to determine if certain personality types are more likely to react positively or negatively to certain disease treatments. We will use a variety of supervised methods (such as regression) to determine if there are associations between personality type and disease treatment sentiment. In addition we want to take advantage of additional forms of metadata, to see if they correlate with either disease treatment or personality, such as profile picture, network connectivity, geolocation, and embedded URLs in comments. In applicable cases we want to be able to distinguish between valid information and "hoaxes" to protect our analyses from biases due to deliberate misinformation.

% This should describe the potential of the proposed activity
% to advance knowledge in the field of mathematics. 
The information produced through data mining and machine learning will be useful to provide clinicians and public health professionals feedback concerning successes and challenges associated with disease treatments. Pertinent tools and summary datasets will also be produced and made available to researchers. In addition novel applications of machine learning and data science methods will be developed as necessary, and shared with the data science community to assist and inspire related applications in other domain areas.