
%%%%%%%%% PROPOSAL -- 15 pages (including Results from Prior NSF Support)

\required{Project Description}

% From the NSF Grants Proposal Guide:
% "The Project Description should provide a clear statement of the work 
% to be undertaken and must include the objectives for the period of the 
% proposed work and expected significance; the relationship of this work
% to the present state of knowledge in the field, as well as to work in 
% progress by the PI under other support.
%
% The Project Description should outline the general plan of work, 
% including the broad design of activities to be undertaken, and, 
% where appropriate, provide a clear description of experimental 
% methods and procedures. Proposers should address 

% what they want to do,
% why they want to do it,
% how they plan to do it,
% how they will know if they succeed,
% and what benefits could accrue if the project is successful.

% The project activities may be based
% on previously established and/or innovative methods and approaches,
% but in either case must be well justified. These issues apply to 
% both the technical aspects of the proposal and the way in which
% the project may make broader contributions."

\section{Mining Pre-Exposure Prophylaxis Trends in Social Media}

Pre-Exposure Prophylaxis (PrEP) is a recently developed method for the prevention of Human Immunodeficiency Virus (HIV) via the administration of an oral pharmaceutical trade named Truvada. Truvada contains active ingredients tenofovir and emtricitabine, both Nucleotide Reverse Transcriptase Inhibitors (NRTIs). In the last four years, since Truvada was approved for PrEP in 2012, PrEP has shown demonstrated efficacy at preventing HIV for HIV negative individuals in serodiscordant relationships\cite{liu2014early}. Though existing methods of effective HIV prevention exist, data shows that they may not be used in case of unexpected sexual contact or personal preference\cite{taylor2016life}. PrEP is also well suited to individuals in socially or economically underprivileged groups, and for the HIV negative partner in serodiscordant couples\cite{ware2012s}. In experimental studies, PrEP has been studied as a method to safely conceive a child\cite{lampe2011achieving}.

Initial studies of PrEP have shown that it is highly effective\cite{golub2013efficacy}, however because it is still a new treatment, it is facing a number of medical and social obstacles before it reaches full adoption. Incomplete clinical and patient knowledge, social stigma, and uncertain insurance status have been identified as challenges preventing continued adoption\cite{calabrese2015stigma}. Also, since Truvada is an oral NRTI, it must be taken daily. Cases where patients do not adhere to their full prescription have led to loss of viral protection, and some patients and clinicians worry that lack of adherence could lead to increased risk of infection with drug resistant strains\cite{arnold2012qualitative}.

% clearly state the goal, previous research, and why our research exceeds previous research
The goal presented in this proposal is to improve PrEP adoption and efficacy by reviewing PrEP patients and other HIV community members perceptions of the challenges and successes that have taken place during PrEP's initial adoption. While local clinical monitoring has uncovered some challenges facing PrEP efficacy and adoption\cite{van2013high}, large scale data mining has the potential to capture a broader perceptive than local clinical reports can. In addition, by mining the massive data sentiments present on social media platforms, researchers are able to capture unfiltered opinions, and can update disease monitoring analyses in real time. Previous work has used Twitter to predict county-level HIV prevalence\cite{young2014methods}, and general HIV discussion monitoring\cite{young2013online}. However to our knowledge, prior to the previous work described in the next subsection, data mining social media specifically to determine thoughts and sentiments surrounding PrEP has not been performed.

\subsection{Previous Work}

Our goal in this previous research was to determine perceptions and sentiments of PrEP using social media, in order to address challenges related to PrEP adoption and efficacy. Though we have investigated the use of multiple social media platforms including Facebook, Grindr, Reddit and Twitter, we have focused primarily on the use of Twitter for past projects and intend to use Twitter for future analyses. Twitter is especially useful because in addition to textual data, various useful metadata is also available including datetime, geolocation, username, hastags, and external hyperlinks. Twitter also features a convenient well documented and free API and has over 300 million monthly active users.

In this analysis we collected over 1 million tweets from the Twitter streaming API filtered on PrEP and HIV related keywords. By using embedding techniques such as Word2Vec and Doc2Vec, we were able to identify new keywords and hashtags that are relevant to the PrEP conversation on Twitter including unexpected political connections "NancyReagan" and popular hashtags associated with PrEP that might not have been known to researchers in advance such as "\#whereisprep" (see table 1 for examples of hashtags and structurally-related tweets). In addition Doc2Vec can be used to query the top N tweets related to a given hashtag or the top N users related to a given hashtag. This allowed us to identify a small subset of the N tweets most structurally related to "PrEP" that can be human-read without requiring humans to read the full dataset. Reading from this set of structurally important tweets, we uncovered important blog articles exploring some of the fears of drug resistant forms of HIV and concerns as to whether these strains could be caused in part by over use or misuse of PrEP medications. The linked blog articles that we found highlight the usefulness of the hyperlink metadata embedded in tweets. Through these hyperlinks, Twitter acts as an index, providing indirect access to a much larger external social media ecosystem.

% Doc2Vec table
\begin{table}
\centering
\caption{Cosine similarity to document-vector ``\#PrEP''}
\begin{tabular}{|l|c|} \hline
Related hashtag/tweet & Cosine similarity to \#PrEP\\ \hline
\#lgbtmedia16 & 0.739128\\ \hline
\#hiv & 	0.727602 \\ \hline
\#whereisprep & 0.707165 \\ \hline
\#truvada & 0.696113 \\ \hline
\#hivprevention & 0.636068 \\ \hline
tweet-702179860983189504 & 0.630055\\ \hline
user-711275699529764864 & 0.629254\\ \hline
tweet-708519265540907010 & 0.628778 \\ \hline
tweet-712032637024653313 & 0.628646 \\ \hline
\#harrogatehour & 0.628547 \\ \hline
\hline\end{tabular}
\label{tbl:d2v}
\end{table}

Our Doc2Vec results also allowed us to identify the top N users most relevant to PrEP. By querying the Twitter REST API for these users' timelines, and using topic modeling methods like Latent Dirichlet Analysis (LDA), we identified the word-distribution-topics present in the Twitter conversation (see figure 1). This analysis went beyond the simple keyword identification analysis from Word2Vec since it clustered keywords into topics and it operated on all tweets that PrEP-related users tweeted, not just PrEP related tweets. The LDA results showed a variety of related concerns such as other sexually transmitted diseases, LGBT related topics, health insurance and political topics. Neither PrEP or Truvada were present in the top 30 keywords related to HIV/AIDS demonstrating that PrEP is still a nascent rare topic in the online discussion. An extension to LDA, Dynamic Topic Modeling (DTM), was able to capture topic and word frequency over time. The DTM results showed that the keyword "PrEP" is increasing in relative frequency over time, even relative to related words such as "pill", "prevention" and "drug". This demonstrates increased interest in PrEP which may correlate with an increase in PrEP adoption over the data acquisition period.

One of the key terms identified through LDA topic modeling was "stigma", a concern that is a known issue impeding PrEP adoption in previous studies\cite{liu2014early}. Stigma is an issue with sexually transmitted diseases in general, including HIV, resulting in many HIV positive indivisuals not knowing their status, and thus may not be aware that they are transmitting the disease when they come in sexual contact with HIV negative individuals. Previous studies have also suggested that stigma may contribute to issues surrounding PrEP adherence\cite{calabrese2015stigma}.

\begin{figure*}
\centering
\includegraphics[height=3in, width=5.0in]{figures/fixFig5}
\caption{LDA topic modeling for the top 500 users related to PrEP.}
\label{fig:lda}
\end{figure*}

Using an open dataset of tweets which were labeled with binary sentiment labels, we performed a semi-supervised classification sentiment analysis. In the sentiment analysis, 5513 tweets were provided by Sanders Analytics. These tweets were human labeled either positive or negative. The sentiment analysis was performed by converting the tweets to numerical vectors using the unsupervised Doc2Vec embedding method. These numerical vectors were then classified according to the sentiment labels, and the trained model was used to infer sentiment labels for each of our PrEP-related tweets.

The sentiment analysis allowed us to identify N PrEP related tweets with the highest sentiment, and N PrEP tweets with the lowest sentiment. After performing the automated sentiment analysis, a human was inserted into the analysis loop to quickly read the top positive and negative tweets to get a sense of the most important successes and concerns present in public PrEP perception. In the positive tweets we found hyperlinks to blogs with positive firsthand accounts from individuals successfully using PrEP to stay HIV negative. In the negative tweets we found concerns of whether Truvada can protect against drug resistant strains of HIV (example negative tweets shown in table 2). Other concerns suggested that over prescription of Truvada could give individuals a false sense of security, and lead to a rise of non-HIV sexually transmitted diseases. These results show that patients, and the public at large need to be educated on the specific risks of drug resistant strains.

The issue of individuals gaining a false sense of security towards other sexually transmitted diseases is being addressed though clinical education for individuals that are prescribed Truvada, though these results show that this information is lacking in some individuals, presumably not taking Truvada. This implies that the dissemination of Truvada educational information through a platform like Twitter may improve overall PrEP-related knowledge and address some of the concerns and stigmas currently associated with PrEP. Together these approaches and results show that we can take raw text and metadata and extract keywords, hastags, temporal trends, and sentiment information. Doc2Vec and sentiment classification allow the researcher to extract a set of the N most highly relevant tweets from a large corpus that can be easily human-readable.

% do a top negative tweets table (also with tweet ID)
% table 4
\begin{table*}
\centering
\caption{Negative sentiment tweets.}
\begin{tabular}{|p{2.5cm}|p{10cm}|} \hline
Category & Text\\ \hline
General & ``Also, how f***ing vile of Hillary to say. Reagan did f***ing NOTHING during the AIDS epidemic until it was too late. What a stupid old hag.''\\ \hline
General & ``I wonder why he beat her a** when she was tryna leave like she wasn't gone be running back when she found out she had HIV \& nobody want her''\\ \hline
General & ``Aaannd. Hillary Clinton breathes a sigh of relief that Twitter has left its outrage of her AIDS comments behind to tend to Drumpf debacle.''\\ \hline

PrEP specific & ``RT gaston\_croupier \#Truvada patent's not expired yet but it is sold online as a generic drug? There's something rotten in internet \#PrEP h''\\ \hline
PrEP specific & ``Equality\_MI Syph \& Hep C have gone up 550\% in Gay Men bc many feel tht bc they're on PrEP, they don't need condoms. HIV isn't the only STI.''\\ \hline
PrEP specific & ``Xaviom8 in interviews he says he was adherent. strain was highly resistant, and Truvada wouldn't have blocked it anyways. PrEP didn't fail.''\\ \hline

Truvada specific & ``not surprised at all that someone got HIV on truvada. people get pregnant on birth control. tomato-condoms are still important-tomahto''\\ \hline
Truvada specific & ``Now reading that truvada does not protect against certain strains of the HIV virus. Yet people want to take that risk..''\\ \hline
Truvada specific & ``I think I have conjunctivitis unless truvada cured it overnight cuz im not feeling as horrible today as last night''\\ \hline

\hline\end{tabular}
\label{tbl:neg}
\end{table*}

Though our previous research on data mining social media has uncovered important tweets, keywords, sentiments and hashtags related to the Twitter discussion of PrEP, many important questions remain poorly understood. One important issue is determining why patients stop adhering to their PrEP medication. While our LDA results uncovered "stigma" and other related keywords, and some of the critical tweets we identified described uncertainty in the efficacy of PrEP, this question still remains to be fully answered. One of the challenges with using Twitter as a data source is that we cant verify personal information for the people authoring the tweets. For example we don't know if they are taking Truvada, or whether they are HIV negative or positive, or other important details of their medical status. We also don't know if misinformation or excessive negativity is being spread by uninformed individuals, or by nefarious individuals. By incorporating medical data we could potentially identify direct connections between sentiment and written opinions with more concrete medical outcomes, though such connections would fall under The Health Insurance Portability and Accountability Act of 1996 (HIPAA) and require patient consent and/or medical board approval. As of the time this previous work was carried out we were unable to access such data.

\subsection{Proposed work}

There are many ways that we can continue to analyze social media data to identify specific issues impeding HIV prevention efforts. One extension we propose is to investigate additional diseases or behaviors that could correlate with risk of developing HIV. One of the overall goals of biosurveilance is to predict disease outbreaks before they happen in order to intervene preventatively. An outbreak in Scott County Indiana in early 2016 was thought to have been caused by drug usage. In a town of 4,000 people 135 people were diagnosed with HIV, and about 80\% of those diagnosed were codiagnosed with hepatitis C. In theory some online social activity may have been able to indirectly identify this outbreak before it happened\cite{conrad2015community}. In order to identify and predict these hot-spots we can make use of the geolocation metadata, and also do network analyses to identify subgroups, and how information and via inference, social interactions propagate through these subgroups. Some first steps along these lines have already been made by our collaborators \cite{young2014methods}, who used drug-related terms to predict HIV prevalence at the county level. 

% use mentions of PrEP to predict Hepatitus prevalence
In order to extend this work, we could use the presence of other disease keywords known to coinfect with HIV as a basis for predicting HIV prevalence. This proposal is facilitated since we already have a Twitter corpus that contains tweets mentioning HIV, hepatitis B and C and about 30 other infectious diseases. Searching the HIV literature shows that the use of Twitter mentions of Hepatitis C (codiagnosed at a rate of 80\% in Scott County\cite{scottcountyhepc}) and/or other STDs to predict HIV prevalence are lacking. Investigation of the Hepatitis-HIV connection is also particularly important because it allows us to test the concern that we identified in our previous work, that overuse, or misuse of PrEP could lead to increased prevalence of Hepatitis C. One way to test and measure this criticism of PrEP, would be to try to predict Hepatitis C prevalence using mentions of PrEP on Twitter at a county level. If this prediction succeeds, then we have evidence that confirms and quantifies this criticism, and can seek to address how PrEP is administered to prevent coinfection with Hepatitis, however if this prediction fails, we have shown that PrEP is indeed not leading to increased levels of Hepatitis, and thus we will have strengthened the case for PrEP adoption. Quantification of the predictability of the Hepatitis C connection would help potential patients and clinicians determine the benefits and risks associated with using PrEP as the primary HIV protection method.

Finally, we would like to mention that our quantitative approaches and computational pipelines for mining qualitative sentiments surrounding disease treatment provide an important contribution by themselves to the larger data science community. In our previous work we have shown a specific application where we mine social sentiment to identify what is working and what the challenges are for PrEP, but a similar framework could easily be taken and applied to improve the social barriers surrounding some other disease treatment like cancer and chemotherapy. Pharmaceutical companies, academic researchers, and hospitals can use our open source code with minimal modification to monitor their disease and treatment of interest to monitor and improve the outcomes and happiness of their patients. We anticipate that the computational approaches produced during the course of the proposed work will also demonstrate useful methods that can be applied to other areas of public health research.


\section{Combining Social Media and Phylogenetic information to infer HIV outbreak dynamics}

Our previous work mentioned in section 1 of this proposal relied exclusively on social media data, which prevented us from quantifying direct medical outcomes of how sentiment affects PrEP usage and HIV protection. Specifically, we uncovered concerns of drug resistant strains of HIV, which PrEP would not provide protection for, and concerns about the spread of other diseases, such as Hepatitus C, as a result of improper use of PrEP. Previous work on small clinical cohorts in San Francisco, 2 Hepatitis C infections among 485 PrEP patients, has quantified the incidence rate at 0.7 per 100 patient years\cite{volk2015incident}. Work based on simulation has predicted that use of PrEP may lead to an increase in drug resistant HIV by 9-40\%\cite{abbas2011factors}. In order to test and quantify these concerns, we need medical data, and in order to be most relevant to our local medical region, that data should come from the north Georgia medical system.

Though we are not aware, from searching the medical and scientific literature, of genomic and epidemiological data being collected from HIV patients in the north Georgia area, we do know of a study conducted in San Diego\cite{little2014using} and Chicago\cite{morgan2017hiv}, that used HIV genomic sequence data from 478 infected individuals to infer a phylogenetic infection network. These researchers also associated epidemiological data, such as age, sex, risk factors, cell count, and viral load with the phylogenetic infection networks, see figure\ref{fig:phylogenetic}. By examining the genomic information, these researchers were able to identify the incidence of drug resistant forms, and how HIV subtypes spread in the local area over the collection period. The SanDiego study examined the affect of anti-retroviral therapy (ART) on the propensity to generate an infection, finding significantly less transmission when ART was started within the first 12 months. The Chicago study found that HIV transmission happened sporadically throughout the city with no correlation to the individual's region of residence implying that transmission encounters happen far from home. Together, these two studies give information on the factors and aspects driving HIV transmission, and the efficacy of certain treatments, though since these studies were performed prior to Truvada's FDA approval in 2012 (1996-2011 and 2005-2011 respectively), there was no investigation of PrEP.

% Phylogenetic network
\begin{figure*}
\centering
\includegraphics[height=3in, width=5.0in]{figures/pylo_network.png}
\caption{Phylogenetic network from San Diego study\cite{little2014using}.}
\label{fig:phylogenetic}
\end{figure*}

In both the Chicago and San Diego studies, the pol region of HIV-1 was sequenced as a routine assessment of potential drug resistance. To generate the pylogenetic networks, these sequences were aligned using multiple sequence alignment, and while different pylogentics software was used in the two papers, both papers used a cutoff of 1.5\% genetic sequence distance to determine an edge connection between individuals. Date of infection was able to determine, in many cases, a directionality for edges in the transmission network.

\subsection{Proposed work}

To our knowledge, no one has yet combined genomic, epidemiological and social media data to infer and quantify the direct medical outcomes of PrEP usage or link social media sentiments surrounding PrEP to direct HIV medical outcomes. Furthermore, we are not aware of any complex HIV network studies being conducted on the population of the northern Georgia. Thus to better understand the effects of PrEP, and quantify the direct risks and concerns surrounding PrEP in the north Georgia region, we propose a project to combine genetic, and epidemiological data with social media data to determine the interactions between PrEP usage, PrEP social media sentiment, and incidence of drug resistant HIV and Hepatitis C.

We will acquire anonymized sequence and epidemiological data for HIV from the north Georgia region, and use standard phylogentic tools to quantify the presence of drug resistant strains and construct a phylogenetic network using methods described in \cite{little2014using}. We will gather the epidemiological and network details, including geolocation trends, individuals gender, race, risk factors and cell counts. A simple linear correlation of all of these variables with keywords from social media, and the corresponding p-value for these correlations, binned by geographical area, would allow us to determine the social media keywords most strongly associated with each of network and epidemiological variables.

% Chicago regions
\begin{figure*}
\centering
\includegraphics[height=3in, width=5.0in]{figures/chicago_regions.png}
\caption{Regions of Chicago from \cite{morgan2017hiv}.}
\label{fig:chicago}
\end{figure*}

One potential issue in this proposal, is how to make connections between individuals' medical information and social media. This is made especially hard since to make such a connection would violate ethical concerns, and also HIPPA privacy laws. How researchers have gotten around this concern is to bin social media and medical patients into geographical regions and then ask what are the differences or correlations between the regions. For example in the Chicago study\cite{morgan2017hiv} the researchers separated the city into regions to compare and determine at-risk locations, see figure\ref{fig:chicago}. This requires that our data be geotagged in some way. Conveniently the Twitter social media that we have is tagged down to the latitude and longitude coordinates, allowing us to bin by county, by city or by state depending on the precision of the geolocation data attached to the medical data.

% talk about serotype

% talk about drug resistent forms of HIV and their genomic features, and their drug treatment requirments

% talk about HAART

% talk about infering pylogenetic maps of infection

% follow the Hawaii Paper for inspiration

% address how we will get data

% address how we will determine in social media between HAART and non-HAART HIV treatments being discussed

% address the methods we will use to determine the drug resistance of the sequencing strains in our data

% address how we will link medical sequencing data and social media data.

In addition to linear correlations by geographical area, we also want to propose some more complicated network methods. One such method, would be to use the infection transmission graph to perform spectral clustering on the geographical regions (we use connections between individuals in different regions to infer connections between geographical regions). This produces a clustering of regions by connectivity. We then ask whether these clustered regions are similar by their social media activity. Hypothesis testing by comparing this clustering to a randomized clustering (performed by simulation) would tell us whether or not the connectivity between regions affects their medical or social media attributes. Intra-regional connectivity is another variable that can be correlated with social media terms in the simple linear correlation proposed above. If successful, this research would demonstrate an underlying mechanism, the network interactions, affecting HIV transmission. It has been hypothesized and many simulations have been done predicting the affect of giving transmission hub, high risk individuals PrEP in an effort to prevent HIV for the overall region and this network analysis could provide an more precise empirical result to calibrate those simulations.

The primary concern that would threaten this plan would be the inability to acquire data, either because it has not been collected in our geographical region, clinicians are unwilling to collaborate, or some regulatory issue prevents us from gaining access to this data. If this happens, and we are unable to get recent HIV sequence data that would match our social media data by collection dates and locations (our PrEP-related social media data was collected nationwide November 2015-ongoing), one solution would be to request social media data from a date range and location for which HIV genetic and epidemiological data is available.

In the scenario that we are still unable to acquire medical and social media data from the same set of dates and geographical location, another thing that we can do is attempt to acquire prescription data instead. Our collaborators have mentioned before that they may be able to acquire prescription data that contains anonymized information on individuals purchasing various prescription drugs. This would not allow us to directly infer transmission networks, or quantify the rate of drug resistant forms of HIV, but it would provide some medical data to associate with social media data. For example we could use the prevalence of HAART therapy as a proxy for drug resistant HIV prevalence, other ART as a proxy for general HIV, and Hepatitis drugs as a proxy for hepatitis. We could link the levels of these prescriptions, with levels of Truvada prescription and social media sentiment keywords over the time period.



\section{Determining associations between personality and drug perception on social media}

Our initial PrEP research mentioned in section 1 of this proposal, and work by other researchers, has demonstrated that social media can be used to monitor and identify issues of drug adherence, usage concerns, and attitudes of patients, potential patients, and other interested parties using social media data. These subjective thoughts and attitudes are important since while they do not capture direct medical outcomes, if patients, potential patients and other interested parties aren't subjectively satisfied with their treatments, this provides an important avenue to guide improvements to the treatment. Personality may also provide some explanation for adoption and adherence to prescriptions, which greatly influences medical outcomes.

In order to further categorize individuals attitudes and perceptions in social media data, researchers have made use of quantitative psychology profiles. The most standard profile, the big five personality score\cite{gosling2003very}, has demonstrated an ability to predict a variety of attitudes and behaviors, and has since been accepted as one of the best currently available model of personality. The big five profile is composed of extraversion, a measure of sociability, agreeableness, a measure of social trust, conscientiousness, a measure of impulse control, neuroticism, a measure of emotional instability, and openness, a measure of adventurousness.

Researchers have used the big five personality profile in conjunction with social media to study addiction\cite{andreassen2013relationships}, and predict suicide rates\cite{jashinsky2014tracking}. In the context of prescription drugs, one study found that extraversion correlated negatively with levels of drug adherence of antidepressants using medical surveys\cite{cohen20045}, though this study did not use social media. The study found that extraversion was found to be a more significant predictor than severity of depression symptoms. Other work has used the big five personality to predict drug and alcohol abuse\cite{livingston2015role} using data collected from surveys posted on social media. This study found that extraversion and conscientiousness were associated with increased drug and alcohol usage.

Recently, studies have used big five personality information in social media data to predict HIV prevalence in a regional geographical area using social media data \cite{ireland2015future}. In this research, the county-level HIV prediction model showed that contentiousness, or future-orientation, associated negatively with HIV prevalence, demonstrating an indirect connection between risk taking language posted online and risk of HIV infection.

Despite these previous efforts, little of the prevailing research has been conducted from the perspective of trying to identify concerns towards prescription drugs in order to understand and improve patient treatment perceptions. Thus the goal proposed in this section aims to build on this research area by using Twitter data to determine associations between a set of popular medications including antidepressent drugs, pain drugs, and STD treatment and prevention drugs. By combining personality and sentiment information, this will allow us to determine aspects of personality that are favorable or unfavorable to disease treatments. We have chosen to focus on these specific drugs in order to build on established connections between mental health drugs and personality types, and also to continue to investigate psychological motivations related to PrEP adoption and adherence.

\subsection{Previous work}

We acquired keywords associated with personality types from previous researchers \cite{schwartz2013toward}. These researchers associated words with personality scores from 75,000 facebook users who took a big five personality test, and published the resulting keywords correlated with each of the five personality categories. For sentiment data, we used a dataset provided by Sanders Analytics that contained tweets that were human-labeled as positive or negative sentiment. We also acquired a set of tweets containing at least one popular psychiatric, pain, or HIV related prescription drug in the top 100 drugs by sales.

We trained a word2vec model on the sentiment and prescription drugs, inferred sentiment labels for the prescription drug tweets, and separated the prescription drug tweets into positive and negative sentiment corpses. Then we constructed heatmaps showing the word2vec similarity between the prescription drug names and psychology keywords. For each of the 5 psychology categories, we used five keywords most positively associated with that category. We constructed heatmaps showing the overall word2vec similarity between keywords, and a heatmap showing the word2vec similarity from the positive corpus minus the similarity from the negative corpus. In both cases the rows (prescription drugs), were clustered using hierarchical clustering. 

For the psychiatric drugs, we found that extraversion, agreeableness and openness terms tended to be associated with most of the psychiatric drugs in the positive sentiment tweets relative to the negative sentiment tweets, while neurotocism tended to have the opposite pattern\ref{fig:psych_heatmap}. This makes sense based on the relative sentiments broadly associated with these psychological categories. In the same heatmap, we found that Pristiq and Strattera, two Serotonin–norepinephrine reuptake inhibitor (SNRI) class drugs\cite{howland2010potential}, clustered closely together, despite commonly being prescribed for different conditions, depression and attention deficit hyperactivity disorder (ADHD )respectively. The other ADHD drug in our dataset, Vyvanase, is an amphetamine class drug and clustered far from Strattera, perhaps demonstrating that these drugs have very different sentiments in the social media discourse. Our data show some evidence that of the two drugs, Strattera is the one with more positive sentiments, especially in the categories of extroversion, agreeableness and openness. Ongoing and future work can uncover further relationships.

% Psychiatric heatmap here
\begin{figure*}
\centering
\includegraphics[height=3in, width=5.0in]{figures/psych_heatmap}
\caption{Relative similarity for psychiatric drug and personality keywords for positive sentiment tweets relative to negative sentiment tweets. The top 5 personality keywords associated with each of the following big 5 categories are shown in order: extraversion, agreeableness, conscientiousness, openness, neuroticism.}
\label{fig:psych_heatmap}
\end{figure*}

The HIV treatment related drugs show a similar high level pattern, with all three drugs disproportionately positive in relation to extraversion, agreeableness and openness terms\ref{fig:hiv_heatmap}. This provides further evidence for a high-level trend relating drug sentiments and these broad psychological categories. The clustering pattern shows Truvada and Atripla, nucleotide reverse transcriptase inhibitors (NRTI's) clustering together, and Norvir, a drug containing a protease inhibitor clustering farther away.  Norvir seems to have a higher overall positive sentiment than the other two HIV drugs. This trend cannot be explained by sentiment concerning side effects, since as a highly active antiretroviral theropy (HAART) Norvir has more sever side effects at typical doses than the other two drugs, and is administered to combat later stages of AIDs infection. It is possible that the increased positive sentiment for Norvir is related to its role as an affordable highly effective drug treatment option. Historically, Norvir has had one of the largest impacts to dramatically reduce deaths from HIV\cite{hogg1997decline}.

% HIV related heatmap here
\begin{figure*}
\centering
\includegraphics[height=3in, width=5.0in]{figures/hiv_heatmap}
\caption{Relative similarity for HIV-related drug and personality keywords for positive sentiment tweets relative to negative sentiment tweets. The top 5 personality keywords associated with each of the following big 5 categories are shown in order: extraversion, agreeableness, conscientiousness, openness, neuroticism.}
\label{fig:hiv_heatmap}
\end{figure*}

We have also done similar work examining the interactions between personality and sentiment towards pain related drugs, but we have omitted it here in order to keep this proposal concise.

\subsection{Proposed work}

Further work will focus on two priorities, validating the high-level sentiment results, and identifying finer details explaining high level patterns. In order to address the issue of validation, we propose performing a similar analysis to the one described in our previous work on other social media data. We have access to a large dataset containing millions of Reddit comments and Reddit has been used in several medically related datamining analyses performed by other researchers\cite{chen2015combining}. Reddit represents a similar social media platform to Twitter, and while it lacks certain metadata, it has a much larger maximum character limitation. This may allow text mined from Reddit to contain more context than Twitter derived text, either way, if an analysis performed on Reddit data provides similar results, this gives us confidence in our Twitter results.

The incorporation of Reddit data for validation could fail, notably due to lack of mentions of the relevant pharmaceutical drugs. In exploratory research months ago, we found Reddit data to be relatively lacking in HIV related terms relative to Twitter. In this case, we could come up with a statistical way of validating the broad patterns shown in our heatmaps above. We could use a do statistical hypothesis testing. Though word2vec has no generally accepted statistical hypothesis testing associated with it, we could perform a sort of statistical hypothesis testing via simulation, to provide corresponding p-values for each value in the heatmap. We could also provide a p-value to the overall interaction of an average of the 5 psychology keywords in each category. Significant p-values would give us some measure of confidence in the high level interactions we are seeing.

The second priority that we have is delving deeper into the details underlying the psychology-drug sentiments we are seeing. Here we may be able to have a human read the small number of tweets that mention both a drug and a psychological keyword of interest. Alternatively we would use a sorting procedure based on doc2vec, as described in our previous work in section 1, to identify the most positive and most negative tweets associated with a given disease-psychological keyword of interest. Investigation of these specific tweets, and the metadata associated with them may be able to explain some of the factors underlying the high-level trends we uncovered in our preliminary results.

If successful, our results will show how psychology influences drug sentiment, which in turn will give us some information on how to improve drug perception. Improved drug perception will allow clinicians and public health professionals to better educate potential patients to the benefits and concerns associated with a drug, leading to better adherence, adoption, and eventually better health outcomes.

%% begin old proposal
%
%For the third section of this proposal, I would like to extend biosurveillance measures to other disease/treatment combinations that were not covered in the PrEP section, and also take into account social network connectivity (number of followers and followees), and other forms of media related to social profiles such as images. Connectivity is important in part due to its association with certain personality traits. Previous research in psychology has determined that certain narcissistic personality traits correlate with connectivity in social networks\cite{buffardi2008narcissism}. In this section we propose an exploration of whether personality traits correlate with disease/treatment discussions in social media.
%
%Because personality traits have been shown previously to correlate with profile pictures and other posted pictures\cite{ong2011narcissism}, for our investigation of personality traits we wanted to use a social media platform in which users would post images of themselves. While Instagram, a social media platform that specializes in images and short videos, seemed like a great choice for this analysis, unfortunately their public API feed was deactivated indefinitely in June 2016. Facebook also does not have a free API that can be used by data mining researchers, in part because its privacy policy limits many comments to only be viewable by friends. These public API limitations left us with Twitter as the only remaining platform with millions of monthly active users who have profile pictures available for datamining. Twitter is also uniquely useful because it offers a variety of meta data including datetime, hashtag and geolocation data. Twitter also offers a public streaming API and various REST APIs that can be used to query all public tweets and unlike social media platforms like Facebook, tweets are public by default. Twitter users each have a profile picture available through the REST API, and images posted in tweets can be easily filtered out based on the image format (i.e. *.jpeg or *.png).
%
%A recent psychology study has annotated over 1 million tweets by Myers-Briggs personality type and gender\cite{plank2015personality}. We propose the use of this labeled tweet corpus to perform supervised label inference on our public health dataset of interest in a similar way to how we did the sentiment analysis in our PrEP-related research (see section one of this document). In this way we can transpose personality labels onto our disease-related tweets.  We can also see if there are any aspects of user profile pictures that correlate with either personality or disease topics, or finally whether follower or followee counts associate with disease or personality traits. In addition using network-based semi supervised learning methods, such as graph-laplacian\cite{sindhwani2005beyond}, we can do unsupervised learning wherein we use unlabeled twitter users's graph connections to help us better classify labeled user's personalities.
%
%For this project I am proposing the use of several tweet corpuses. Firstly we have the tweets with labeled personality score and gender provided as an open dataset\cite{plank2015personality}. Next we have a corpus of general tweets that mention an infectious disease, a corpus that mentions at least one of the top 100 prescription pharmaceuticals, and a corpus that mentions at least one of the top 100 hospitals in the US. In addition to these disease related corpuses, I also want to repeat any analyses on a random "background" corpus of general tweets from Twitter to make sure that trends and patterns identified in disease related corpuses are disease-specific and not trends that hold over all tweets.
%
%I want to use a variety of simplistic models such as linear regression and decision trees to determine the features that contribute to whether a tweet is associated with a certain personality or mentions a certain disease negatively or positively. These simple models are useful because they can more easily be interpreted. Next I want to do a variety of hypothesis tests to determine if certain diseases or personality traits correlate significantly with each other. This correlation will have to take into account whether or not the disease or personality scores are continuous or discrete, though this will be decided on at the time. Significance tests will use the background tweet corpus described above as an empirical "null distribution".
%
%Taking additional advantage of metadata available in tweets, we can determine if there are any correlations or relevant summary statistics that correlate between disease or personality variables. Given two personality types or two sets of positive or negative disease-related tweets, we can measure the most frequent domain name prefixes in the hyper links. We can also measure if there are non-random temporal or spacial (using geolocation metadata) distribution of these tweets relative to some background distribution (hence the need for a set of background tweets).
%
%A final goal that we want to explore is the phenomenon of hoaxes or fake news. With the advent of the internet and various social media platforms, news can spread quickly from unvetted sources. This has produced an issue of misinformation, including misinformation that is deliberatly spread to gain clicks or spread discrediting information. Some research has shown that spread of hoaxes can be modeled using ecological infection models\cite{tambuscio2015fact}. Recent research has used supervised machine to classify reviews as authentic or fake\cite{banerjee2015using} by using simple models (logistic regression, decision tree, naive bayes, etc.). It may be possible to use more complicated models, such as deep learning models and network-based models, to identify fake news in twitter public health data. This will likely require acquisition of a dataset that has labeled tweets of positive or negative news. If successful, this work would be significant on its own, and it would also help ensure our other biosurvelience research is resistant to fake news biases.
%
%If successful, this research would help public health officials determine the interactions between social media network connectivity, infectious disease, and personality traits. This would continue to inform our understanding developed from research described in section 1 that attempts to determine the social aspects relevant to disease treatment outcomes.

% \required{Broader Impacts}
% As in the project summary, broader impacts of the proposed work
% must be called out separately in the project description.  
% You may be able to give more specific examples, 
% or discuss how you've previously achieved these impacts.
% It should be similar, but not identical, to the Broader Impacts statement
% in the project summary.