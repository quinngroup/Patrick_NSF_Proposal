
%%%%%%%%% BIOGRAPHICAL SKETCH -- 2 pages

\required{Biographical Sketch: Patrick Breen}

% Your Bio should be divided into the following sections
% It is not required to include the parenthecital letters preceding
\required{(a) Professional Preparation}
% This is your educational background:
% Undergrad, Location, Major, Year
% Graduate, Location, Major, Year
% Postdoc, Location, Area, Years-Inclusive
Bowdoin College BA Biochemistry and Mathematics, Brunswick Maine,  2013 \\
University of Georgia PhD Bioinformatics, Athens Georgia, (attending)
\required{(b) Appointments}
% List most recent first
Oak Ridge National Laboratory Fellowship (2016) \\
President of Bioinformatics Graduate Student Association (2015) \\
University of Georgia Presidential Graduate Fellowship (2013)

\required{(c) Products}
% List up to 5 related to the proposal, and up to 5 "Other Significant Products"
% Must be citable and accessible.
% Including, but not limited to: publications, data sets, software, patents and copyrights.
% Unacceptable product examples are: unpublished documents
% (not yet submitted for publication) and invited lectures.
% Each product must include full citation information.
% Including (as applicable and available) names of authors,
% date of publication or release, title, title of enclosing work such
% as journal or book, volume, issue, pages, website and URL.
% If only publications are included, you may use the header "Publications"
Mining Pre-Exposure Prophylaxis Trends in Social Media. Patrick Breen, Jane Kelly, Timothy Heckman, Shannon Quinn. DSAA2016. \\
P2Y6 receptor antagonist, MRS2578, inhibits neutrophil activation and aggregated NET formation induced by gout-associated monosodium urate crystals. Payel Sil ... Patrick Breen ...
\required{(d) Synergistic Activities}
% List up to 5 eaxamples
% Per NSF guidelines, they should "demonstrate the broader impact 
% of the individual's professional and scholarly activities 
% that focuses on the integration and transfer of knowledge
% along with its creation".
% Examples include, but not limited to: innovations in teaching and training
% (i.e. development of curricular materials and pedagogical methods);
% contributions to the science of learning; development and/or refinement 
% of research tools; computation methodologies, and algorithms for 
% problem-solving; development of databases to support research 
% and education; broadening the participation of groups underrepresented in STEM; 
% and service to the scientific and engineering community outside 
% of the individual’s immediate organization.)
Interacted with the local scientific community by judging at high school science fair (2014)


